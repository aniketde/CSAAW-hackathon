Some classical statistical techniques have been used to study the LBNL dataset. For instance, in [main paper], the basic approach consists of using weighted square differences between the chemical composition of two artifacts to determine their similarity, which is then used to cluster the collection of artifacts. This clustering can then be compared to the actual locations of the different sites where the artifacts were found.

The clustering of artifacts according to their chemical composition can say much about the relation between their regions of origin. For instance, if two regions contain many artifacts falling in the same chemical composition cluster, it follows that the materials used in the construction of the artifacts were similar, and perhaps that the two regions shared resources or had markets that were highly connected.

There are some issues with the classical approaches that we wish to attack. Mainly, there is no clear way to analyze connectivity among different archaeological sites based on the clustering of artifacts. A clustering of artifacts based on chemical composition that mimics well the geographical location of the archaeological sites is very illustrative, but it is not obvious how it can be used to quantify the degree of similarity between different sites.

Our contribution to the analysis of the LBNL data is to provide a well-justified similarity measure between archaeological sites based on the chemical composition of their respective artifacts. We are then able to provide quantitative description of the amount of connectivity between different archaeological sites.
