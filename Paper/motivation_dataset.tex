\subsection{Motivation and Summary Stats}
Archaeological studies has been actively inviting the analytical methods from
the entire spectrum of social science studies. However, when it comes to the
application of machine-learning techniques, we have not seen much done recently.
Therefore, we set out to explore the archaeological databases, searching for
datasets that are rich enough, in terms of dimensions of the observables.

Baring the capability of machine-learning in mind, we found the 
Lawrence Berkeley National Laboratory (LBNL) Nuclear Archaeology Program
Archives %\cite?? [the archive]
of particular interest, as it not only contains geographical and historical
records of the artifacts, it also carry the chemical compositions for most of
its records.


Due to limited time and resources, we focused exclusively on the Greece records,
totaling 886 pieces of artifacts from ancient Greece, discovered from 31
archaeological sites\footnote{30 in Greece, and one (Perati in Attica) in
Turkey.}.  The time frame in our sample covers from the Early Helladic era
(around 3200 BC) to Roman Republic. 

\subsection{Data manipulation}
For the records of artifacts from Greece, the dataset came with 1,198
observations in the begining, pulled directly from \verb|http://core.tdar.org/|.
The variables include the discovery site, the associating era, geo-coordinates
and the chemical compositions of 33 elements\footnote{Elements are:
    \textit{Al Ca V Dy Mn Na K Sr As U Eu Ba Sm La Ti Lu Nd Co Sc Fe Ce Yb Cs Ta
    Sb Cr Th Ni Rb Tb Hf Zn}
}. However, the chemical composition
data is not complete. Thus, we dropped a subset of observations as well as
insignificant variables of elements (\textit{Sb Ba As Sr V}) to obtain a sample
of 886 observations. 


