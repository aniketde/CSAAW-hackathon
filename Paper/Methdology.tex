\section{Methodology}

\label{sec:methodology}

%822,27,20

We are interested in analyzing artifacts from $N=20$ archaeological sites. We denote by $X^i$ the collection of artifacts in the $i$-th site, that is $X^i=\{x^i_1,x^i_2,\dots,x^i_{n_i}\}$, where $n_i$ is the number of artifacts in site $i$ and $x^i_m$ refers to the $m$-th artifact in the $i$-th site.

The chemical composition data available for each artifact consists of the concentration of $L=27$ different chemicals. Artifact $x^i_m$ then consists of the (normalized) vector containing these chemical concentrations.

The simplest way to compare the chemical composition of two artifacts is through their square distance

$$ \| x^i_m - x^j_l \|^2 = \sum_{k=1}^L ( (x^i_m)_k - (x^j_l)_k )^2$$

where $(x^i_m)_k$ refers to the $k$-th (normalized) chemical concentration of artifact $x^i_m$.

This distance is taken to represent reasonably when two artifacts are similar. If their distance is small they are considered more similar than if their distance is large. The square distance can serve as the basis to construct many other similarity measures, for instance

$$ k(x^i_m,x^j_l) = \exp(-h\|x^i_m - x^j_l \|^2)$$

where $h>0$ is a parameter. This corresponds to the widely used Gaussian kernel. Other similarity measures $k$ between artifacts based on their chemical composition are possible. The question of which similarity measure between artifacts based on chemical compositions is more appropriate for statistical inference is an important one, but we do not consider it further.

The challenge is to construct a similarity measure between different sites based on the similarities ${k(x^i_m,x^j_l)}$ between their respective artifacts. We propose constructing such similarity function by looking at the average similarity between all pair of artifacts $(x^i_m,x^j_l)$ belonging to sites $X^i$ and $X^j$, respectively, given by

$$S(X^i,X^j) = \frac{1}{n_i n_j} \sum_{m=1}^{n_i} \sum_{l=1}^{n_j} k(x^i_m,x^j_l).$$

There is a theoretical justification for this similarity measure between archaeological sites, given by the theory of kernel mean embedding. This theory will be described in Section~\ref{sec:kme}.

We now wish to illustrate how well the similarity measure $S$ performs in finding similarities between archaeological sites based on the chemical composition of their respective artifacts. We do this by providing a network plot of the archaeological sites, where two sites $X^i$ and $X^j$ are considered connected if their similarity $S(X^i,X^j)$ is larger than a prescribed threshold. The results are illustrated in the figures~\ref{fig:whatevs}.

\section{Similarity}

\label{sec:similarity}

In this section we detail the way in which similarities are computed. We draw inspiration from a well known set of methods in the machine learning community, and acknowledge that the mathematical machinery might not be familiar to the intended audience. Therefore the reader is invited to skip this section on the first read and come back to it when necessary.

We think of the chemical composition vectors $\{x^i_m\}_{m=1}^{n_i}$ corresponding to the artifacts in the $i$-th site $X^i$ as consisting of samples from some distribution $P_i$.

\subsection{Reproducing Kernels}

\label{kernels}

To build a network we need an appropriate notation of similarity between sites (probability densities). We desire the probability densities to inhabit a well behaved space of functions (a Hilbert Space) where a common notion of similarity exists. We propose to embed each pd into the so called RKHS of a kernel K and use iner product in this space as the similarity measure.

The similarities we propose capture similarities among probability densities as elements of a function space. To determine these similarities we need first determine the function space the probability densities inhabit. The function space we will use is the so called Reproducing Kernel Hilbert Space (RKHS) associated to a reproducing kernel. We now define these mathematical objects.

A {\it reproducing kernel} is a symmetric function $k: \sx\times \sx\rightarrow \mathbb{R}$ for which there exists a unique Hilbert space of functions $\sh$ such that $k(\cdot,x)\in \sh$ for all $x\in \sx$ and the reproducing property

\begin{align}

f(x) = \left<f,k(\cdot,x)\right>

\end{align}

holds for all $f\in\sh$ and all $x\in \sx$

\subsection{Kernel Mean Embedding}

\label{kme}

Let $k$ be a reproducing kernel as in section, the {\it kernel mean embedding} of the probability density $P$ in the RKHS $H$ of $k$ is

$$\phi_0(P) = \int k(\cdot,x)P(x) dx$$

Since the true form of $P$ is unknown we use the available data to estimate the KME. Let ${x_l}_{l=1}^n$ be an iid sample from $P$, then the empirical kme of $O$ is

$$\phi(P)=\frac{1}{n}\sum_{l=1}^n k(\cdot,x_l).$$

Although $\phi(P)$ depends on the sample we will omit this dependencies to simplify notation.

With this tool at hand, we define the similarity between site $i$ and site $j$ as the inner product of their corresponding KME's.

\begin{align*}

S(X^i,X^j)&=\langle \phi(P_i),\phi(P_j) \rangle_H \\

&= \frac{1}{n_i n_j} \sum_{l=1}^{n_j} \sum_{m=1}^{n_j} \langle k(\cdot,X_l^i),k(\cdot,x_m^j) \rangle_H \\

&= \frac{1}{n_i n_j} \sum_{l,m} k(x^i_l,x^j_m)

\end{align*}

where the last equality follows from the reproducing property.
